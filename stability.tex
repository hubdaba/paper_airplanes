
%% This is an example first chapter.  You should put chapter/appendix that you
%% write into a separate file, and add a line \include{yourfilename} to
%% main.tex, where `yourfilename.tex' is the name of the chapter/appendix file.
%% You can process specific files by typing their names in at the 
%% \files=
%% prompt when you run the file main.tex through LaTeX.
\section{Stability}

\subsection{Laminar to Turblent Transition}

Because of the low Reynold number of paper airplane aerodynamics, we have to worry about
the persistance of the laminar boundary layer.
The Reynold number of paper airplane flight is on the order of $10^4$.
There is a big difference between turbulent flow and laminar boundary
layers. Laminar boundary layer doesn't respond well to an increasing
pressure gradient in the direction of flow, whereas turbulent flow does.
However, turbulent flow results in large frictional effects and can result in stalling.

One aspect of the laminar boundary layer is the location in which it separates from the surface.
When the laminar boundary layer seperates from the surface, this results Kelvin
Helmholtz instability, vortices, and finally a turbulent boundary layer. 
Because of the instability this transition between turbulence and laminar flow 
can cause, predicting this location is important in the stability of the flight. 

This transition location also results in hysterisis effects. Before separation
occurs, increasing angle of attack will increase lift. But afterwards,
decreasing the angle of attack does not return to the same lift. This effect is
illustrated in figure~\ref{fig:boundary_layer_transition}.

The transition location is characterized as a separation bubble when the laminar layer
separates from the surface. After a short distance the turulent layer would reattach.
At high Reynold numbers the transition region is very short, but for lower Reynold
numbers, this distance could be long and the turbulent layer might not have the chance
to reattach. 

One effect of laminar vs turbulence is the skin friction. When the flow is turbulent there
is a lot more skin friction then when it is laminar. The other side of it is, 
in terms of lift, maximum
lift depends on the characteristics of the turbulent boundary layer, because the
turbulent layer has the ability to recover pressure at the
rear of the airfoil. 

\begin{figure}[hl]
  \centering
    \includegraphics[scale=.5]{figures/boundary_layer_transition.png}
    \caption{Illustration of the boundary layer process}
  \label{fig:boundary_layer_transition}
\end{figure}



\begin{enumerate}
\item For Reynold numbers between 1000 and 10000 the boundary layer flow is laminar
and doesn't transition into trubulent flow. This type of flight is seen with
the dragon fly and the house fly. 
\item  For Reynold numbers between 10000 and 30000 the boundary layer is laminar but
it can separate. When it separates it does not reattach.
\item At 30000 to 70000 there are hysterisis effects because of the transition. 

\end{enumerate}

\subsection{Lateral Stability}

\subsubsection{Dihedral vs Anhedral}

Dihedral and anhedral refer to the angle between the wings and the horizontal plane.
Dihedral wings means that the wings point upwards, and the anhedral wings means the wings point
downwards as shown in Figure~\ref{fig:dihedraleffect}. Dihedral wings have more lateral 
stability than anhedral wings.

\begin{figure}[hl]
  \centering
    \includegraphics[scale=.5]{figures/dihedraleffect.png}
    \caption{Dihedral vs Anhedral wings}
  \label{fig:dihedraleffect}
\end{figure}

The bank angle is defined as the angle about the axis along
the plane's motion. Let us call the bank angle $\phi$. When the aircraft is
banked, this results in a rolling moment, called the dihedral effect. If the
rolling moment opposes the bank, the dihedral effect is positive. In order for
there to be stability, the dihedral effect needs to be positive. There are many
different factors that can contribute to the dihedral effect: the shape of the wings,
the wing placement, tail height, etc. 

To analyze this effect in more detail we will show that a dihedral aircraft has 
a positive dihedral effect. When the aircraft is banked we see that there is a net force
to the side, this is called sideslip (Figure~\ref{fig:dihedral1}). 
However we see that the forces on the wings do not 
contribute to the dihedral effect yet.


\begin{figure}[hl]
  \centering
    \includegraphics[scale=.5]{figures/dihedral1.png}
    \caption{SideSlip}
  \label{fig:dihedral1}
\end{figure}

At this point, with respect to the plane's frame of reference, there is air flowing 
to the side that will contribute to a dihedral effect. However, we see that for the
corresponding anhedral plane, the bottom wing would be affected more by the air, causing
a negative dihedral effect (Figure~\ref{fig:dihedral2}). If the bank angle is $\phi$ and
the angle that the wings make with the horizontal is $\theta$ then after factoring in
the banked angle, the right wing has an angle of attack of  $\theta - \phi$.
The left wing has an angle of attack $- \theta - \phi$.

\begin{figure}[hl]
  \centering
    \includegraphics[scale=.5]{figures/dihedral2.png}
    \caption{Positive vs Negative dihedral effect}
  \label{fig:dihedral2}
\end{figure}

If we assume that the wings are approximate flat plates, we can say that the
coefficient of lift is approximately $2\pi\sin(\alpha)$ as shown in Equation~\ref{flat_plate_cl}. 
$\alpha$ is
the angle of attack and $\beta$ is a constant. For small angles it is 
approximately $2\pi(\alpha)$. So the left wing will exerience a downward force
and the right wing will experience an upwards force, straightning the plane out.


