%% This is an example first chapter.  You should put chapter/appendix that you
%% write into a separate file, and add a line \include{yourfilename} to
%% main.tex, where `yourfilename.tex' is the name of the chapter/appendix file.
%% You can process specific files by typing their names in at the 
%% \files=
%% prompt when you run the file main.tex through LaTeX.
\chapter{Low Aspect Ratio Wings at Low Reynold Numbers}

Low aspect ratio wings are wings where the ratio between the wing span and
the chord is small. Commertial aircrafts typically have high aspect ratio wings
to increase the amount of lift on the aircraft, but there are benefits of having
low aspect wings. Low aspect ratio aircrafts need to deal with less bending
stress because the wings are shorter. Also, the drag due to surface friction is
smaller because of the lower Reynold number because of the bigger lengthscale.
For paper airplanes, low aspect ratio wings are definitely better than high
aspect ratio wings because there is no way for paper to overcome the
bending stress.


\section{Previous Research}

A lot of research about low aspect ratio aircrafts were done in the form of
micro-air vechicle research. This has the most relevance with paper airplanes because
micro-air vehicles operate on a Reynold number regime similar to paper airplanes. 
In 1999, Mueller showed that cambered plates offer better performance, and
trailing-edge geometry does not have a strong effect on lift and drag
for thing wings at low Reynolds numbers.

\section{Stability}

The low Reynold number of paper airplane aerodynamics contributes a lot towards the
stability of the flight.
The Reynold number of paper airplane flight is on the order of $10^4$.
Therefore, variations of airfoil characteristics will result in rapid changes in 
the stability. At low Reynold numbers, the laminar boundary layer has great 
influence. There is a big difference between turbulent flow and laminar boundary
layers. Laminar boundary layer doesn't respond well to an increasing
pressure gradient in the direction of flow, whereas turbulent flow does.
However, turbulent flow results in large frictional effects.

One aspect of the laminar boundary layer is when it separates from the surface.
Above we stated that it seperates under adverse pressure gradients. When the
laminar boundary layer seperates from the surface, this results Kelvin
Helmholtz instability, vortices, and finally a turbulent boundary layer. 
Because of the instability this transition between turbulence and laminar flow 
can cause, predicting this location is important in the stability of the flight.
This transition location also results in hysterisis effects.
